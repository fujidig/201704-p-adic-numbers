\documentclass[uplatex]{jsarticle}
\usepackage{amsmath}
\usepackage{amssymb}
\usepackage{amsthm}
\usepackage{framed}

\newcommand{\Z}{\mathbb{Z}}
\newcommand{\Q}{\mathbb{Q}}
\newcommand{\R}{\mathbb{R}}

\theoremstyle{definition} %斜体にしない
\newtheorem{thm}{定理}
\newtheorem{defi}[thm]{定義}
\newtheorem{lem}[thm]{補題}
\newtheorem{prop}[thm]{命題}
\newtheorem*{remark*}{注意}
\newtheorem*{example*}{例}
\renewcommand{\proofname}{証明}


\begin{document}

\title{$p$進数の初歩}
\maketitle
%\pagebreak

\section{導入}

次の式を見てください。

\begin{align*}
-1 &= 1 + 2 + 2^2 + 2^3 + 2^4 + \dots \\
-1/2 &= 1 + 3 + 3^2 + 3^3 + 3^4 + \dots \\
\sqrt{-1} &= 2 + 5 + 2 \cdot 5^2 + 5^3 + 3 \cdot 5^4 + 4 \cdot 5^5 + \dots
\end{align*}

実数や複素数として考えるとこの式は間違いです。なぜなら右辺は $+\infty$ に発散するからです。
しかし、$p$進数の世界ではこれらは完全に正しい式です。一つ目は$2$進数、二つ目は$3$進数、三つ目は$5$進数として成り立ちます。

$p$進数とはなんでしょうか。一言でいえば$p$進数の全体$\Q_p$は有理数全体$\Q$を$p$進距離と呼ばれる距離によって完備化した空間であり、それに四則演算の構造を入れたものです。ここに$p$は素数です。

つまり、各素数$p$ごとに実数とは異なる数の世界$\Q_p$があるのです。各$\Q_p$は$\Q$を含んでいます。
\[
\Q_2, \Q_3, \Q_5, \Q_7, \Q_{11}, \dots \supset \Q \subset \mathbb{R}
\]

これから$\Q_p$を定義し、その性質を見ていきます。この発表が終わる頃には皆さんに最初の3つの式を納得していただけるだろうと期待しています。

\section{$p$進数の定義}

$p$進数を定義するには、$p$進距離を導入する必要があります。そこでまず、数学でいう距離とは何なのか定義します。

\begin{oframed}\begin{defi}[距離空間]
$X$を集合, $d$を$X \times X$から$\R$への写像とする。任意の$x, y, z \in X$について次を満たすものとする。
\begin{enumerate}
	\item $d(x, y) \geq 0$
	\item $d(x, y) = 0 \iff x = y$
	\item $d(x, y) = d(y, x)$
	\item $d(x, z) \leq d(x, y) + d(y, z)$
\end{enumerate}
このとき、$d$を$X$上の距離関数という。

また$(X, d)$あるいは単に$X$を距離空間という。
\end{defi}\end{oframed}

たとえば、$\R$で$d(x, y) = |x - y|$と定義すると$(\R, d)$は距離空間です。確認しましょう。この距離関数$d$をユークリッド距離といいます。
ユークリッド距離関数を有理数に制限した関数も$d$と書くことにすると$(\Q, d)$も距離空間です。

同じ集合でも距離の入れ方は一つとは限りません。
実際、これから$\Q$に$|x - y|$とは異なる$p$進距離$d_p$を定めます。

以下$p$を素数とします。

\begin{oframed}\begin{defi}[$p$進付値]
整数$n \ne 0$の素因数分解を
\[n = \pm p_1^{e_1} p_2^{e_2} \dots p_k^{e_k}\]
とする。このとき$n$の$p$進付値を
\[v_p(n) = \begin{cases}e_i & (\exists i, p = p_i) \\ 0 & (\text{otherwise}) \end{cases} \]
で定める。

有理数$x = m/n\,(m,n \in \Z, m, n \ne 0)$の$p$進付値を
\[v_p(x) = v_p(m) - v_p(n)\]
で定める。(これはwell-definedである)
\end{defi}\end{oframed}

\begin{oframed}\begin{defi}[$p$進絶対値]
有理数$x \ne 0$の$p$進絶対値を
\[|x|_p = p^{-v_p(x)}\]
で定める。$0$に対しては
\[|0|_p = 0\]
と定める。
\end{defi}\end{oframed}

\begin{oframed}\begin{defi}[$p$進距離]
有理数$x, y$の$p$進距離を
\[d_p(x, y) = |x - y|_p\]
で定める。
\end{defi}\end{oframed}

$p$進距離が距離関数になることは確認する必要があります。

しかし、その前に$p$進距離がどういう距離か説明します。
$p$進距離は$p$でたくさん割れれば割れるほど$0$に近いという距離です。
実際、
\begin{align*}
d_2(1, 0) &= 1 \\
d_2(2, 0) &= 1/2 \\
d_2(4, 0) &= 1/4 \\
d_2(8, 0) &= 1/8 \\
\vdots
\end{align*}
となります。

\begin{oframed}\begin{prop}\label{prop:1}
任意の$a, b \in \Q$について
\begin{enumerate}
\item $v_p(ab) = v_p(a) + v_p(b)$
\item $v_p(a+b) \geq \min\{v_p(a), v_p(b)\}$
\end{enumerate}
\end{prop}\end{oframed}
\begin{proof}
あとで
\end{proof}

\begin{oframed}\begin{prop}\label{prop:2}
任意の$a, b \in \Q$について
\begin{enumerate}
\item $|a|_p \geq 0$
\item $|a|_p = 0 \iff a = 0$
\item $|ab|_p = |a|_p |b|_p$
\item $|a+b|_p \leq \max\{|a|_p, |b|_p\}$
\end{enumerate}
\end{prop}\end{oframed}
\begin{proof}
1と2は明らかである。3と4は命題 \ref{prop:1}から従う。
\end{proof}

命題 \ref{prop:2}の4を$p$進絶対値の非アルキメデス性といいます。

\begin{oframed}\begin{thm}
$d_p$は$\Q$上の距離関数である。
\end{thm}\end{oframed}
\begin{proof}
あとで
\end{proof}

$\Q_p$は$\Q$を$p$進距離によって完備化した空間だと述べました。そこで完備化を説明する必要があります。そのために、点列の収束、コーシー列、完備性という概念を説明します。

\begin{oframed}\begin{defi}[点列の収束]
$(X, d)$を距離空間、$\{x_n\}_{n\in\mathbb{N}}$をXの点の列とする。
このとき$\alpha \in X$が存在して
\[
\forall \epsilon>0, \exists N\in\mathbb{N}, \forall n \in \mathbb{N}, n \geq N \Rightarrow d(x_n, \alpha) < \epsilon
\]
を満たすとき、$X$の点列$\{x_n\}_{n\in\mathbb{N}}$は$\alpha$に収束するといい、
\[
\lim_{n \to \infty} x_n = \alpha
\]
と書く。
\end{defi}\end{oframed}

\begin{oframed}\begin{defi}[コーシー列]
$(X, d)$を距離空間、$\{x_n\}_{n\in\mathbb{N}}$をXの点の列とする。
\[
\forall \epsilon>0, \exists N\in\mathbb{N}, \forall n \in \mathbb{N}, n \geq N \Rightarrow d(x_n, x_N) < \epsilon
\]
を満たすとき、$\{x_n\}_{n\in\mathbb{N}}$はコーシー列であるという。
\end{defi}\end{oframed}

定義より収束列は明らかにコーシー列です。

\begin{example*}
$X$の任意の距離空間、$x \in X$を任意の点としたとき点列$\{x, x, x, \dots\}$つまり$x_n = x \,(\forall n \in \mathbb{N})$は$x$に収束する。
実際、任意の$\epsilon > 0$に対して$N = 1$とすれば$n \geq 1$のとき$d(x_n, x) = d(x, x) = 0 < \epsilon$。
特にこの点列はコーシー列である。
\end{example*}

\begin{example*}
$\R$をユークリッド空間、$x_n = 1/n$とするとこれは$0$に収束する。
実際、任意の$\epsilon > 0$に対して$N$を$1 / \epsilon$より大きい自然数とおけば、$n \geq N$のとき$d(1/n, 0) = 1/n \leq 1/N < \epsilon$。特にこの点列はコーシー列である。
\end{example*}

\begin{example*}
$\R$をユークリッド空間、$x_n = n$とするとこれはコーシー列ではない。
実際、任意の$N \in \mathbb{N}$に対して$n = N+1$とおけば$d(x_N, x_{n}) = |N - (N+1)| = 1 \geq 1$。特にこの点列は収束列でない。
\end{example*}

\begin{example*}
$d_p$を$p$進距離として距離空間$(\Q, d_p)$を考えると、$x_n = p^n$は$0$に収束する。
実際、任意の$\epsilon > 0$に対して$p^{-N} < \epsilon$となるよう$N$をとると、$d_p(p^n, 0) = p^{-n} \leq p^{-N} < \epsilon$となる。
\end{example*}

\begin{example*}
$d$をユークリッド距離として距離空間$(\Q, d)$を考え、点列
\begin{align*}
a_1 &= 1 \\
a_2 &= 1.4 \\
a_3 &= 1.41 \\
a_4 &= 1.414 \\
a_5 &= 1.4142 \\
\vdots
\end{align*}
を考える。この点列はコーシー列だが収束列ではない。(空間を$\R$とすれば当然$\sqrt 2$に収束する)
\end{example*}

\begin{oframed}\begin{defi}[完備]
$(X, d)$を距離空間とする。$X$の任意のコーシー列が収束するとき$X$は完備であるという。
\end{defi}\end{oframed}

たとえば$d$をユークリッド距離とすると$(\R, d)$は完備ですが、$(\Q, d)$は完備ではありません。

\begin{oframed}\begin{defi}[稠密性]
$(X, d)$を距離空間とする。$S \subset X$が次の条件を満たすとき、$S$は$X$において稠密であるという。
\[
\forall x \in X, \forall \epsilon > 0, \exists y \in S, d(x, y) < \epsilon
\]
\end{defi}\end{oframed}

完備ではない距離空間も、それに点を付け加えて完備にすることができます。すなわち次が成立します。

\begin{oframed}\begin{thm}[完備化]
$(X, d)$を距離空間とする。このとき距離空間$(\tilde{X}, \tilde{d})$であって次を満たすものがある。
\begin{enumerate}
\item $X \subset \tilde{X}$
\item $\tilde{d}$の$X \times X$への制限は$d$に等しい
\item $\tilde{X}$は完備である
\item $X$は$\tilde{X}$において稠密である
\end{enumerate}
$(\tilde{X}, \tilde{d})$を$(X,d)$の完備化という。
\end{thm}\end{oframed}
\noindent 証明の概略。
$X$のコーシー列全体を$C(X)$と書き、$C(X)$に次の同値関係を入れる。

\[\{x_n\} \sim \{y_n\} \iff \forall \epsilon > 0, \exists N \in \mathbb{N}, \forall n \in \mathbb{N}, n \geq N \Rightarrow d(x_n, y_n) < \epsilon\]

そして$\tilde{X} = C(X)/{\sim}$とおく。また
\[\tilde{d}(\{x_n\}, \{y_n\}) = \lim_{n\to\infty} d(x_n, y_n)\]
とおく。このとき$(\tilde{X}, \tilde{d})$が定理の条件を満たす。 

なお、$X$の元$x$にコーシー列$\{x, x, x, \dots\} \in C(X)$を対応させることで$X$を$\tilde{X}$の部分集合だとみなす。 (証明の概略終わり)

実は、$(\Q, d)$を完備化したものが$(\R, d)$なのであります。

\begin{oframed}\begin{defi}
$(\Q, d_p)$の完備化を$(\Q_p, \tilde{d_p})$と書く。
\end{defi}\end{oframed}

これで距離空間としての$\Q_p$は定義されました。$\Q_p$の元のことを$p$進数といいます。しかし四則演算がまだ定義されていません。四則演算(のうち和・差・積)は次のように定義します。

\begin{oframed}\begin{defi}
$\Q_p$に和・差・積を次のように定義する。

$\tilde{x}, \tilde{y} \in \Q_p$に対して$\Q$の点列$\{x_n\}, \{y_n\}$で$x_n \to \tilde{x}, y_n \to \tilde{y} (n \to \infty)$となるものをとる。このとき
\begin{align*}
\tilde{x} + \tilde{y} &= \lim_{n\to\infty} (x_n + y_n) \\
\tilde{x} - \tilde{y} &= \lim_{n\to\infty} (x_n - y_n) \\
\tilde{x} \tilde{y} &= \lim_{n\to\infty} (x_n y_n)
\end{align*}
と定める。
\end{defi}\end{oframed}

これらはwell-definedです。
たとえば和の場合、$\{x_n\}, \{y_n\}$が収束するとき$\{x_n + y_n\}$も収束します。また、$\tilde{x}, \tilde{y}$に対してそれに収束する$\{x_n\}, \{y_n\}$の取り方は無数にありますが、その取り方によらず$\lim_{n\to\infty} (x_n + y_n)$の値は定まります。

こうして定義した演算に関して$\Q_p$は環になります。

\begin{oframed}\begin{defi}[環]
集合$A$とその上の二項演算$+$, $\cdot$について、元$0, 1 \in A$が存在して次を満たすとき$(A, +, \cdot)$あるいは単に$A$を環という。

任意の$x, y, z \in A$に対して
\begin{enumerate}
\item $0 + x = x$
\item $\exists x'\in A, x + x' = 0$
\item $x + (y + z) = (x + y) + z$
\item $x + y = y + x$
\item $1 \cdot x = x$
\item $x \cdot (y \cdot z) = (x \cdot y) \cdot z$
\item $x \cdot y = y \cdot x$
\item $x \cdot (y + z) = x \cdot y + x \cdot z$
\end{enumerate}
\end{defi}\end{oframed}

$\Q_p$は環であるだけではなく体でもあります。

\begin{oframed}\begin{defi}[体]
集合$K$が環であって次を満たすとき$K$を体という。

任意の$x \in A, x \ne 0$に対して$x \cdot x' = 1$となる$x' \in K$がある。
\end{defi}\end{oframed}

\begin{oframed}\begin{thm}
$\Q_p$は体である。
\end{thm}\end{oframed}
\begin{proof}
環であることは$\Q$が環であることと、$\Q_p$の演算の定義より出る。
たとえば結合法則$\tilde{x} + (\tilde{y} + \tilde{z}) = (\tilde{x} + \tilde{y}) + \tilde{z}$は$Q$の点列$\{x_n\}, \{y_n\}, \{z_n\}$で$x_n \to \tilde{x}, y_n \to \tilde{y}, z_n \to \tilde{z} (n \to \infty)$となるものをとるとき
\begin{align*}
\tilde{x} + (\tilde{y} + \tilde{z}) &= \lim_{n\to\infty} (x_n + (y_n + z_n)) \\
 &= \lim_{n\to\infty} ((x_n + y_n) + z_n) \\
 &= (\tilde{x} + \tilde{y}) + \tilde{z}
\end{align*}
というように示せる。

任意の$0$でない元$\tilde{x}$について$\tilde{x'}$が存在して$\tilde{x} \tilde{x'} = 1$となることは、$\tilde{x'} = \lim_{n\to\infty}{1/x_n}$とおけば示せる。(ただしこの極限が存在することを示す必要がある)
\end{proof}

$\Q$に定義されていた$p$進絶対値を$\Q_p$に拡張します。

\begin{oframed}\begin{defi}
$\tilde{x} \in \Q_p$に対して
\[|\tilde{x}|_p = \tilde{d_p}(\tilde{x}, \tilde{0})\]
と定める。
\end{defi}\end{oframed}

\begin{oframed}\begin{prop}\label{prop:4}
任意の$\tilde{a}, \tilde{b} \in \Q_p$について
\begin{enumerate}
\item $|\tilde{a}|_p \geq 0$
\item $|\tilde{a}|_p = 0 \iff \tilde{a} = \tilde{0}$
\item $|\tilde{a}\tilde{b}|_p = |\tilde{a}|_p  |\tilde{b}|_p$
\item $|\tilde{a}+\tilde{b}|_p \leq \max\{|\tilde{a}|_p, |\tilde{b}|_p\}$
\end{enumerate}
\end{prop}\end{oframed}
\begin{proof}
1と2は定義より出る。3と4は命題 \ref{prop:2}の3と4の極限をとればよい。
\end{proof}

$\Q_p$において、和、差、積、商は連続写像になります。このことの証明は省きます。

\begin{oframed}\begin{defi}
$\Z_p = \{\tilde{x} \in \Q_p \mid |\tilde{x}|_p \le 1 \}$と定め、$\Z_p$の元を$p$進整数という。
\end{defi}\end{oframed}

命題 \ref{prop:4}より$0, 1$は$p$進整数であり$p$進整数同士の和、差、積は$p$進整数となります。よって$\Z_p$は環となります。

なお「$p$進数」と「$p$進法」は異なる概念であることに注意しましょう。$p$進数は$\Q_p$の元のことですが、$p$進法は実数を$p$を底として小数表示することを言います。

\section{$p$進数の性質}

以降、$\tilde{d_p}$をチルダなしの$d_p$で書き、$\Q_p$の元も$\tilde{x}$ではなく$x$のような文字を使います。

今、等比数列の和の公式より
\[(1-p^{n+1}) / (1-p) = 1 + p + p^2 + \dots + p^n\]
です。$\Q_p$において両辺を$n \to \infty$とすると$p^{n+1} \to 0$なので、
\[1 / (1-p) = 1 + p + p^2 + \dots\]
となります。
ここに$p = 2, 3$を代入すると最初の3つの式のうちの二つ
\begin{align*}
-1 &= 1 + 2 + 2^2 + 2^3 + 2^4 + \dots (\text{in } \Q_2) \\
-1/2 &= 1 + 3 + 3^2 + 3^3 + 3^4 + \dots (\text{in } \Q_3)
\end{align*}
が示せました!

非アルキメデス性から次の重要な事実が出ます。

\begin{oframed}\begin{prop}\label{prop:3}
$\{a_n\}$を$\Q_p$の点列とする。このとき
\[\Sigma_{n=1}^\infty a_n が収束 \iff \lim_{n\to\infty} a_n = 0\]
\end{prop}\end{oframed}
\begin{proof}
$\Rightarrow$は$\R$のときと同じ証明でよい。

$\Leftarrow$) $|a_{m+1} + a_{m+2} + \dots a_n|_p \leq \max\{|a_{m+1}|_p,  |a_{m+2}|_p, \dots, |a_n|_p\}$より出る。
\end{proof}

命題 \ref{prop:3}から次の事実が従います。

\begin{oframed}\begin{prop}
$k \in \Z$, $a_k, a_{k+1}, \dots \in \Z$, $0 \le a_i < p-1$のとき級数
\begin{align}
\Sigma_{i=k}^\infty a_i p^i \label{eq:1}
\end{align}
は$\Q_p$において収束する。
\end{prop}\end{oframed}

$k=0$である場合、$\Sigma_{i=0}^\infty a_i p^i$を$(\dots a_3 a_2 a_1 a_0)_{(p)}$とも書きます。たとえば $-1 = (\dots 1 1 1 1)_{(2)}$。

$\Q_p$の元が式 (\ref{eq:1})のように表せるとき、その表示を$p$進展開といいます。

$\Q$の元は$p$進展開可能であることを見ましょう。

$a$を$p$と互いに素な整数とします。このとき初等整数論(あるいは初等群論)の事実より
\[p^k \equiv 1 \pmod a\]
となる$k \in \mathbb{N}$が存在します。すると
\[p^k-1 = ab\]
です。これを変形すると
\begin{align*}
\frac{1}{a} &= \frac{-b}{1-p^k} \\
&= -b(1+p^k+p^{2k}+\dots+p^{nk}+\dots)
\end{align*}
となります。$-b$も$p$進展開して積を計算すれば$1/a$の$p$進展開が得られます。

\begin{example*}
$1/5$の$2$進展開を求める。
$2^4-1 = 15 = 3 \cdot 5$である。

\begin{align*}
\frac{1}{5} &= \frac{-3}{1-2^4} \\
&= -3 (1 + 2^4 + 2^8 + \dots+ 2^{4n} + \dots) \\
&= -(1+2) (1 + 2^4 + 2^8 + \dots+ 2^{4n} + \dots) \\
&= -(1 + 2^1 + 2^4 + 2^5 + 2^8 + 2^9 + \dots) \\
&= -(\dots 00110011)_{(2)} \\
&= (\dots 11001101)_{(2)} \\
&= 1 + 2^2 + 2^3 + 2^6 + 2^7 + 2^{10} + 2^{11} + \dots
\end{align*}
\end{example*}

この例から任意の$\Q$の元は$p$進展開可能であることが推察できます。

実は次の事実が知られています。
\begin{oframed}\begin{thm}
任意の$\Q_p$の元($\ne 0$)は一意的に$p$進展開可能。
つまり任意の$x \in \Q_p- \{0\}$について$k \in \Z$, $a_k, a_{k+1}, \dots \in \Z$, $0 \le a_i < p-1$, $a_k \ne 0$が一意的に存在して
\[x = \Sigma_{i=k}^\infty a_i p^i\]
となる。このとき$|x|_p = p^{-k}$。
\end{thm}\end{oframed}
証明はしません。証明の鍵となるものは
\begin{itemize}
\item $p$進絶対値の離散性
\item $\Q$が$\Q_p$において稠密であること
\end{itemize}
です。

\section{ヘンゼルの補題}

\begin{oframed}\begin{thm}[ヘンゼルの補題]
$F(x)$を整数係数の多項式、$x_0 \in \Z$とする。
$\delta_1 = v_p(F(x_0)), \delta_2 = v_p(F'(x_0))$とおく。
もし、$\delta_1 > 2 \delta$ならば、$x \in \Z_p$があり$F(x)=0$。
\end{thm}\end{oframed}
証明はしない。

この$x$は次のように構成される:
\[x_{n+1} = x_n - \frac{F(x_n)}{F'(x_n)}\]
として
\[x = \lim_{n\to\infty}x_n.\]

しかし、この構成法では、近似値$x_n$は有理数である。近似値$x_n$が整数となるような構成法もあり、次のようにする。
\begin{align*}
F'(x_n) &= p^{\delta_2} y_n\\
y_n z_n &\equiv 1 \pmod p\\
x_{n+1} &= x_n - \frac{F(x_n)}{p^{\delta_2}} z_n
\end{align*}
として
\[x = \lim_{n\to\infty}x_n.\]
特に$x_n$の$p$進展開は$x$の$p$進展開と$\delta_1 - \delta_2 + n$桁まで一致する。

ヘンゼルの補題を使って、$F(x) = x^2+1$の根(すなわち$\sqrt{-1}$)が$\Z_5$の中にあることを確認して、その$p$進展開を求めましょう。

$F(x) = x^2+1$より、$F'(x) = 2x$。
$x_0 = 2$とおけば、$F(2) = 5, F'(2) = 4$より、$\delta_1 = v_5(F(2)) = 1, \delta_2 = v_5(F'(2)) = 0$。よって定理の仮定を満たします。

$y_0 = 4, z_0 = -1$であり、$x_1 = 2 - 5 \cdot (-1) = 7 = (12)_{(5)}$。

$y_1 = 14, z_0 = -1$であり、$x_2 = 7 - 50 \cdot (-1) = 57 = (212)_{(5)}$。

$y_2 = 114, z_0 = -1$であり、$x_3 = 57 - 3250 \cdot (-1) = 3307 = (101212)_{(5)}$。

$y_2 = 6614, z_0 = -1$であり、$x_3 = 3307 - 10936250 \cdot (-1) = 10939557 = (10300031212)_{(5)}$。

よって$\sqrt{-1} = (\dots31212)_{(5)}$です。実際に筆算して確かめましょう。

\begin{thebibliography}{9}
\item 彌永昌吉・彌永健一『集合と位相Ⅱ』岩波書店
\item 雪江明彦『整数論1 初等整数論からp進数へ』日本評論社
\end{thebibliography}

\end{document}
